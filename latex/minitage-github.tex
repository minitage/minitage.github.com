% Generated by Sphinx.
\def\sphinxdocclass{report}
\documentclass[letterpaper,10pt,english]{sphinxmanual}
\usepackage[utf8]{inputenc}
\DeclareUnicodeCharacter{00A0}{\nobreakspace}
\usepackage[T1]{fontenc}
\usepackage{babel}
\usepackage{times}
\usepackage[Bjarne]{fncychap}
\usepackage{longtable}
\usepackage{sphinx}


\title{Minitage github Documentation}
\date{December 29, 2010}
\release{Last updated: Dec 29, 2010}
\author{kiorky}
\newcommand{\sphinxlogo}{}
\renewcommand{\releasename}{Release}
\makeindex

\makeatletter
\def\PYG@reset{\let\PYG@it=\relax \let\PYG@bf=\relax%
    \let\PYG@ul=\relax \let\PYG@tc=\relax%
    \let\PYG@bc=\relax \let\PYG@ff=\relax}
\def\PYG@tok#1{\csname PYG@tok@#1\endcsname}
\def\PYG@toks#1+{\ifx\relax#1\empty\else%
    \PYG@tok{#1}\expandafter\PYG@toks\fi}
\def\PYG@do#1{\PYG@bc{\PYG@tc{\PYG@ul{%
    \PYG@it{\PYG@bf{\PYG@ff{#1}}}}}}}
\def\PYG#1#2{\PYG@reset\PYG@toks#1+\relax+\PYG@do{#2}}

\def\PYG@tok@gd{\def\PYG@tc##1{\textcolor[rgb]{0.63,0.00,0.00}{##1}}}
\def\PYG@tok@gu{\let\PYG@bf=\textbf\def\PYG@tc##1{\textcolor[rgb]{0.50,0.00,0.50}{##1}}}
\def\PYG@tok@gt{\def\PYG@tc##1{\textcolor[rgb]{0.00,0.25,0.82}{##1}}}
\def\PYG@tok@gs{\let\PYG@bf=\textbf}
\def\PYG@tok@gr{\def\PYG@tc##1{\textcolor[rgb]{1.00,0.00,0.00}{##1}}}
\def\PYG@tok@cm{\let\PYG@it=\textit\def\PYG@tc##1{\textcolor[rgb]{0.25,0.50,0.56}{##1}}}
\def\PYG@tok@vg{\def\PYG@tc##1{\textcolor[rgb]{0.73,0.38,0.84}{##1}}}
\def\PYG@tok@m{\def\PYG@tc##1{\textcolor[rgb]{0.13,0.50,0.31}{##1}}}
\def\PYG@tok@mh{\def\PYG@tc##1{\textcolor[rgb]{0.13,0.50,0.31}{##1}}}
\def\PYG@tok@cs{\def\PYG@tc##1{\textcolor[rgb]{0.25,0.50,0.56}{##1}}\def\PYG@bc##1{\colorbox[rgb]{1.00,0.94,0.94}{##1}}}
\def\PYG@tok@ge{\let\PYG@it=\textit}
\def\PYG@tok@vc{\def\PYG@tc##1{\textcolor[rgb]{0.73,0.38,0.84}{##1}}}
\def\PYG@tok@il{\def\PYG@tc##1{\textcolor[rgb]{0.13,0.50,0.31}{##1}}}
\def\PYG@tok@go{\def\PYG@tc##1{\textcolor[rgb]{0.19,0.19,0.19}{##1}}}
\def\PYG@tok@cp{\def\PYG@tc##1{\textcolor[rgb]{0.00,0.44,0.13}{##1}}}
\def\PYG@tok@gi{\def\PYG@tc##1{\textcolor[rgb]{0.00,0.63,0.00}{##1}}}
\def\PYG@tok@gh{\let\PYG@bf=\textbf\def\PYG@tc##1{\textcolor[rgb]{0.00,0.00,0.50}{##1}}}
\def\PYG@tok@ni{\let\PYG@bf=\textbf\def\PYG@tc##1{\textcolor[rgb]{0.84,0.33,0.22}{##1}}}
\def\PYG@tok@nl{\let\PYG@bf=\textbf\def\PYG@tc##1{\textcolor[rgb]{0.00,0.13,0.44}{##1}}}
\def\PYG@tok@nn{\let\PYG@bf=\textbf\def\PYG@tc##1{\textcolor[rgb]{0.05,0.52,0.71}{##1}}}
\def\PYG@tok@no{\def\PYG@tc##1{\textcolor[rgb]{0.38,0.68,0.84}{##1}}}
\def\PYG@tok@na{\def\PYG@tc##1{\textcolor[rgb]{0.25,0.44,0.63}{##1}}}
\def\PYG@tok@nb{\def\PYG@tc##1{\textcolor[rgb]{0.00,0.44,0.13}{##1}}}
\def\PYG@tok@nc{\let\PYG@bf=\textbf\def\PYG@tc##1{\textcolor[rgb]{0.05,0.52,0.71}{##1}}}
\def\PYG@tok@nd{\let\PYG@bf=\textbf\def\PYG@tc##1{\textcolor[rgb]{0.33,0.33,0.33}{##1}}}
\def\PYG@tok@ne{\def\PYG@tc##1{\textcolor[rgb]{0.00,0.44,0.13}{##1}}}
\def\PYG@tok@nf{\def\PYG@tc##1{\textcolor[rgb]{0.02,0.16,0.49}{##1}}}
\def\PYG@tok@si{\let\PYG@it=\textit\def\PYG@tc##1{\textcolor[rgb]{0.44,0.63,0.82}{##1}}}
\def\PYG@tok@s2{\def\PYG@tc##1{\textcolor[rgb]{0.25,0.44,0.63}{##1}}}
\def\PYG@tok@vi{\def\PYG@tc##1{\textcolor[rgb]{0.73,0.38,0.84}{##1}}}
\def\PYG@tok@nt{\let\PYG@bf=\textbf\def\PYG@tc##1{\textcolor[rgb]{0.02,0.16,0.45}{##1}}}
\def\PYG@tok@nv{\def\PYG@tc##1{\textcolor[rgb]{0.73,0.38,0.84}{##1}}}
\def\PYG@tok@s1{\def\PYG@tc##1{\textcolor[rgb]{0.25,0.44,0.63}{##1}}}
\def\PYG@tok@gp{\let\PYG@bf=\textbf\def\PYG@tc##1{\textcolor[rgb]{0.78,0.36,0.04}{##1}}}
\def\PYG@tok@sh{\def\PYG@tc##1{\textcolor[rgb]{0.25,0.44,0.63}{##1}}}
\def\PYG@tok@ow{\let\PYG@bf=\textbf\def\PYG@tc##1{\textcolor[rgb]{0.00,0.44,0.13}{##1}}}
\def\PYG@tok@sx{\def\PYG@tc##1{\textcolor[rgb]{0.78,0.36,0.04}{##1}}}
\def\PYG@tok@bp{\def\PYG@tc##1{\textcolor[rgb]{0.00,0.44,0.13}{##1}}}
\def\PYG@tok@c1{\let\PYG@it=\textit\def\PYG@tc##1{\textcolor[rgb]{0.25,0.50,0.56}{##1}}}
\def\PYG@tok@kc{\let\PYG@bf=\textbf\def\PYG@tc##1{\textcolor[rgb]{0.00,0.44,0.13}{##1}}}
\def\PYG@tok@c{\let\PYG@it=\textit\def\PYG@tc##1{\textcolor[rgb]{0.25,0.50,0.56}{##1}}}
\def\PYG@tok@mf{\def\PYG@tc##1{\textcolor[rgb]{0.13,0.50,0.31}{##1}}}
\def\PYG@tok@err{\def\PYG@bc##1{\fcolorbox[rgb]{1.00,0.00,0.00}{1,1,1}{##1}}}
\def\PYG@tok@kd{\let\PYG@bf=\textbf\def\PYG@tc##1{\textcolor[rgb]{0.00,0.44,0.13}{##1}}}
\def\PYG@tok@ss{\def\PYG@tc##1{\textcolor[rgb]{0.32,0.47,0.09}{##1}}}
\def\PYG@tok@sr{\def\PYG@tc##1{\textcolor[rgb]{0.14,0.33,0.53}{##1}}}
\def\PYG@tok@mo{\def\PYG@tc##1{\textcolor[rgb]{0.13,0.50,0.31}{##1}}}
\def\PYG@tok@mi{\def\PYG@tc##1{\textcolor[rgb]{0.13,0.50,0.31}{##1}}}
\def\PYG@tok@kn{\let\PYG@bf=\textbf\def\PYG@tc##1{\textcolor[rgb]{0.00,0.44,0.13}{##1}}}
\def\PYG@tok@o{\def\PYG@tc##1{\textcolor[rgb]{0.40,0.40,0.40}{##1}}}
\def\PYG@tok@kr{\let\PYG@bf=\textbf\def\PYG@tc##1{\textcolor[rgb]{0.00,0.44,0.13}{##1}}}
\def\PYG@tok@s{\def\PYG@tc##1{\textcolor[rgb]{0.25,0.44,0.63}{##1}}}
\def\PYG@tok@kp{\def\PYG@tc##1{\textcolor[rgb]{0.00,0.44,0.13}{##1}}}
\def\PYG@tok@w{\def\PYG@tc##1{\textcolor[rgb]{0.73,0.73,0.73}{##1}}}
\def\PYG@tok@kt{\def\PYG@tc##1{\textcolor[rgb]{0.56,0.13,0.00}{##1}}}
\def\PYG@tok@sc{\def\PYG@tc##1{\textcolor[rgb]{0.25,0.44,0.63}{##1}}}
\def\PYG@tok@sb{\def\PYG@tc##1{\textcolor[rgb]{0.25,0.44,0.63}{##1}}}
\def\PYG@tok@k{\let\PYG@bf=\textbf\def\PYG@tc##1{\textcolor[rgb]{0.00,0.44,0.13}{##1}}}
\def\PYG@tok@se{\let\PYG@bf=\textbf\def\PYG@tc##1{\textcolor[rgb]{0.25,0.44,0.63}{##1}}}
\def\PYG@tok@sd{\let\PYG@it=\textit\def\PYG@tc##1{\textcolor[rgb]{0.25,0.44,0.63}{##1}}}

\def\PYGZbs{\char`\\}
\def\PYGZus{\char`\_}
\def\PYGZob{\char`\{}
\def\PYGZcb{\char`\}}
\def\PYGZca{\char`\^}
% for compatibility with earlier versions
\def\PYGZat{@}
\def\PYGZlb{[}
\def\PYGZrb{]}
\makeatother

\begin{document}

\maketitle
\tableofcontents
\phantomsection\label{index::doc}



\chapter{What is minitage}
\label{index:what-is-minitage}\label{index:welcome-to-minitage-s-documentation}
Minitage is all about deployment and projects packaging


\section{Why should i use minitage?}
\label{why::doc}\label{why:why-should-i-use-minitage}
Is minitage for me : \textbf{Yes, it makes no doubt.}


\subsection{Minitage leitmotivs}
\label{why:minitage-leitmotivs}\begin{itemize}
\item {} 
It makes deployments easy

\item {} 
It enables you to have reproducible environments

\item {} 
Industrialization is now possible

\item {} 
Make it the hard way once, reproduce with just a command a dozens of identical environments.

\item {} 
It save developers with compilation and configuration steps

\item {} 
It save admins with crazy developers stuff

\item {} 
It is completly free (\code{GPL2})!

\end{itemize}

If you have some doubts about them, feel free to inform us on the bug tracker :)


\subsection{A professional tool}
\label{why:a-professional-tool}
It s a collection of tools used by professionals which tends to be well documented and tested.

\href{http://www.makina-corpus.com}{Makina Corpus} use it as its primary mean to deploy websites and intranets.

Some public references:
\begin{itemize}
\item {} 
\href{http://www.guerir.fr/}{http://www.guerir.fr/} (Plone)

\item {} 
\href{http://www.bisonvert.net/}{http://www.bisonvert.net/} (Django + GIS)

\item {} 
\href{http://www.alfa-aci.com/}{http://www.alfa-aci.com/} (Plone, LDAP)

\end{itemize}


\section{About}
\label{about:about}\label{about::doc}

\subsection{Purpose}
\label{about:purpose}\begin{itemize}
\item {} 
Minitage is a meta packages manager.

\item {} 
It's goal is to integrate build systems / other package manager together to make them install in a `well known' layout.

\item {} 
In other terms, it install its stuff in `prefix' and it targets the total isolation from the host system.

\item {} 
Moreover, this tool will make you forget compilation and other crazy stuff that put your mind away from your real project needs.

\item {} 
Another subsidiary goal is to standardize installations, and make development environments as similar as possible with production deployments

\item {} 
With all the precedings targets achieved, minitage would be called a good \code{industrialization tool} :)

\end{itemize}


\subsection{Pre requisite knowledge}
\label{about:pre-requisite-knowledge}\begin{itemize}
\item {} 
Read carefully this documentation: \href{http://plone.org/documentation/tutorial/buildout}{http://plone.org/documentation/tutorial/buildout}.

\item {} 
And this one can be good too: \href{http://pypi.python.org/pypi/zc.buildout}{http://pypi.python.org/pypi/zc.buildout}

\item {} 
It is essential for you to know the basics of buildout to use it.

\item {} 
It is not necessary for you to test it because you will do the practise part just after ;).

\end{itemize}


\subsection{Variables used in this documentation}
\label{about:variables-used-in-this-documentation}\begin{itemize}
\item {} 
\code{\$project} -\textgreater{} your project name

\item {} 
\code{\$bd} -\textgreater{} buildout directory

\item {} 
\code{\$mt} -\textgreater{} minitage root path

\item {} 
\code{\$url} -\textgreater{} the url of your versioned project

\end{itemize}


\subsection{Minitage, the origins}
\label{about:minitage-the-origins}

\subsubsection{Buildout Limits}
\label{about:buildout-limits}
First of all, do not buy a gun before using buildout
Buildout is not a magical tool and we just had quite a lot of drawbacks :
\begin{itemize}
\item {} 
Today offline mode can be problematic

\item {} 
If you change your python, take a coffee.

\item {} 
1GB per project, it s too much.

\item {} 
Buildout behaviour can be hard to predict even more if the configuration file is huge.

\end{itemize}


\subsubsection{The Idea}
\label{about:the-idea}\begin{itemize}
\item {} 
The idea is to write a light and simple package manager from scratch which will allow us to integrate various build systems.

\item {} 
What about shell scripts, buildout or makefiles interact together to assemble all the parts of our project parts?

\item {} 
To allow this, the approach is to share a common `well known' layout to install things in minitage.

\end{itemize}


\subsubsection{Implementation goals}
\label{about:implementation-goals}\begin{itemize}
\item {} 
Write just small specialized buildouts.

\item {} 
Reuse with svn:externals or others (mercurial forest extension, package manager fetching methods)

\item {} 
Centralize and re-use dependencies among projects.

\item {} 
Use different ways to install, One tool for one thing, remember the Unix way.

\item {} 
Deploy a project from start to end.

\item {} 
Isolate all the needed boilerplate from the host system. All stuff in minitage is supposed to be independant from the host base system. Compiled stuff is interlinked as much as possible.

\item {} 
Reproduce the same environement everywhere (on UNIX platforms). It will probably work on
\begin{itemize}
\item {} 
Linux

\item {} 
MacOSX but at least OSX Leopard is required. (it has worked, but not tested recently)

\item {} 
FreeBSD (it has worked, but not tested recently)

\end{itemize}

\end{itemize}


\section{Specifications}
\label{spec:specifications}\label{spec::doc}

\subsection{The package manager in 7 points}
\label{spec:the-package-manager-in-7-points}\begin{itemize}
\item {} 
Be simple !

\item {} 
Run on popular Unixes, including FreeBSD, Linux and MacOSX. And in a second time, why not port him to other OSs.

\item {} 
Have a simple and robust dependency system.

\item {} 
Integrates other build systems

\item {} 
Fetch from several methods: main are svn, http, ftp, Mercurial, Git and Bzr

\item {} 
Provide some system which allow us search for packages in alternate locations.

\item {} 
Install each part independently to the others. Follow the http rule (DO NOT SHARE):

\begin{Verbatim}[commandchars=@\[\]]
Independently is not a synonym of not using another dependency.
It s just to say that you can not install in the same /dependencies/foo for 2 different dependencies at the same time.
You must install each dependency separatly from the others and then reference in other packages.
Be warned that there is not sandbox mecanism for checks and it is not goiong to appear, minitage must be simple.
\end{Verbatim}

\end{itemize}


\subsection{The minilays}
\label{spec:the-minilays}

\subsubsection{Abstract}
\label{spec:abstract}\begin{itemize}
\item {} 
A minilay is a directory containing minibuilds.

\item {} 
The name come from gentoo overlays system. See \href{http://www.gentoo.org/proj/en/overlays/userguide.xml}{http://www.gentoo.org/proj/en/overlays/userguide.xml}

\end{itemize}


\subsubsection{Overlaping minilays}
\label{spec:overlaping-minilays}\begin{itemize}
\item {} 
Minilays are parsed during the dependencies calculation done by minimerge.
minimerge searches for minibuilds in all its minilays, in alphabetical order.

\item {} 
That `s why when you do that:

\begin{Verbatim}[commandchars=@\[\]]
minilays/
    a@_override@_dependencies/
        minibuild1
    dependencies/
        minibuild1
\end{Verbatim}

\item {} 
And try to:

\begin{Verbatim}[commandchars=@\[\]]
minimerge minibuild1
\end{Verbatim}

\end{itemize}

minimerge will install \code{a\_override\_dependencies/minibuild1} instead of \code{dependencies/minibuild1}.

This mechanism is very useful to override things for special needs !


\subsubsection{Specifying alternate minilays to minimerge}
\label{spec:specifying-alternate-minilays-to-minimerge}\begin{itemize}
\item {} 
You can give minilays to minimerge by placing them in:

\begin{Verbatim}[commandchars=\\\{\}]
\PYG{n}{minitage}\PYG{o}{/}\PYG{n}{minilays}
\end{Verbatim}

\item {} 
You can put paths in the MINILAYS environment variable (space separated paths), but be aware, they are read first in the dependencies calculation!:

\begin{Verbatim}[commandchars=@\[\]]
export MINILAYS="path1 path2"
\end{Verbatim}

\end{itemize}


\subsection{The minibuilds}
\label{spec:the-minibuilds}

\subsubsection{Abstract}
\label{spec:id1}\begin{itemize}
\item {} 
They are the minitage packages.

\item {} 
A minibuild is similar to a gentoo ebuild  (i recommend you to read \href{http://en.wikipedia.org/wiki/Ebuild}{http://en.wikipedia.org/wiki/Ebuild}) or a FreeBSD Port Makefile (\href{http://en.wikipedia.org/wiki/Ports\_collection}{http://en.wikipedia.org/wiki/Ports\_collection})  or a macport.

\item {} 
The forms is a `Config.ini' like file.

\item {} \begin{description}
\item[{In a short, it is just a metadata file which contains all the necessary to describe the install process of a dependency:}] \leavevmode\begin{itemize}
\item {} 
Dependencies for the package ? ( eg: python-2.4 )

\item {} 
Where to get it ? ( the url )

\item {} 
How fetch it ? (svn, git, hg, http, ftp)

\item {} 
How will it be installed ? ( buildout )

\item {} 
Metadatas:
\begin{itemize}
\item {} 
License

\item {} 
Project homepage

\end{itemize}

\end{itemize}

\end{description}

\end{itemize}


\subsubsection{Writing Minibuilds}
\label{spec:writing-minibuilds}\begin{itemize}
\item {} 
A minibuild is a config.ini file  which is read by minimerge and has a bunch of metadata :
\begin{itemize}
\item {} 
a src\_uri variable: where to fetch the package if not present on the file system

\item {} 
a src\_type variable: how to fetch it.
Available fetch methods are:
\begin{itemize}
\item {} 
svn

\item {} 
hg

\item {} 
bzr

\item {} 
git

\item {} 
static (for http, file, local and ftp)

\end{itemize}

\item {} 
a dependencies variable: list of other minibuilds which this one depends on

\item {} 
a install\_method variable : how to install it

Available install methods are:
\begin{itemize}
\item {} 
buildout (use the {\hyperref[spec:buildout-maker]{\emph{buildout maker}}})

\end{itemize}

\item {} 
a category variable : controls the category of the minibuilds. {[}{[}br{]}{]}
Categories are top level  directories in your minitage directory. {[}{[}br{]}{]}
Possible categories may be:
\begin{itemize}
\item {} 
dependencies

\item {} 
eggs

\item {} 
misc

\item {} 
django

\item {} 
tg

\item {} 
pylons

\item {} 
zope

\end{itemize}

\item {} 
You can use of course your own categories but be aware to name them
with only letters and digits.

\item {} 
minitage installs a minibuild called MINIBUILDNAME to your minitage/CATEGORY/MINIBUILDNAME.

\item {} 
The steps ran are:
\begin{itemize}
\item {} 
Fetch:
\begin{itemize}
\item {} 
in online mode only: Try to fetch the src\_uri to minitage/CATEGORY/MINIBUILDNAME

\item {} 
in offline mode: The files must be present !

\end{itemize}

\item {} 
Run the install method

\end{itemize}

\end{itemize}

\end{itemize}

Exemple : the minibuilds/cyrus-sasl-2.1 minibuild:

\begin{Verbatim}[commandchars=@\[\]]
@PYGZlb[]minibuild@PYGZrb[]
@# depends on the freetype-2.1 minibuild
dependencies=freetype-2.1
@# must be fetched from svn
src@_type=svn
@# where it is !
src@_uri=https://subversion.makina-corpus.net/zopina/buildouts/buildout-meta/trunk/lib/cyrus-sasl-2.1.22/
@# this is a "buildout" minibuild which will install itself via buildout
install@_method="buildout"
@# will be installed in minitage/dependencies/packagename
category=dependencies
homepage=http://chuknorris.is.a.good/guy
license=GPL
\end{Verbatim}

You must place your minibuild in a minilay.


\subsubsection{Conventions}
\label{spec:conventions}
They are rules, not just conventions, follow them or be killed.
If you want a full review, just take a look to \code{minitage.core/minitage/core/objects/minibuild.py{}`} regular expressions ;).
Please not that your project name cannot contain ``\code{-}'' as it is used to generate eggs names.
\begin{itemize}
\item {} 
Valid names:
\begin{itemize}
\item {} 
meta-toto

\item {} 
double-toto

\item {} 
toto

\item {} 
test-1.0

\item {} 
test-test-1.0

\item {} 
test-1.0.3

\item {} 
test-1.0\_beta444

\item {} 
test-1.0\_py2.4

\item {} 
test-1.0\_py2.5

\item {} 
test-1.0\_beta444\_pre20071024

\item {} 
test-1.0\_alpha44

\item {} 
test-1.0\_alpha44\_pre20071024

\item {} 
test-1.0\_pre20071024

\item {} 
test-1.0\_branchBRANCHNAME

\item {} 
test-1.0\_branchHEADDIGIT

\item {} 
test-1.0\_tagHEADDIGIT

\item {} 
test-1.0\_r1

\item {} 
test-1.0\_rHEAD

\item {} 
test-1.0\_rTIP

\end{itemize}

\item {} 
Please set the version to MAJOR:MINOR without the revision for compiled dependencies !
This will prevent from recompiling everything on security updates for example

\item {} 
Multiple maker specific metadata

\end{itemize}


\subsubsection{The buildout maker}
\label{spec:the-buildout-maker}\label{spec:buildout-maker}
It will run a buildout somewhere

Specific options:
\begin{itemize}
\item {} \begin{description}
\item[{buildout\_config:}] \leavevmode
configuration file  to run

\end{description}

\end{itemize}


\subsection{The installation layout}
\label{spec:the-installation-layout}

\subsubsection{Abstract}
\label{spec:id2}
\begin{Verbatim}[commandchars=@\[\]]
etc/
     minimerge.cfg
bin/
    minimerge
    python
lib/
    python-ver/
        site-packages/
            minitage.core

dependencies/
    dep1/
        buildout.cfg
        hooks/
        patches/::
        parts/
            part/
                bin/
                lib/
                include/

eggs/
    cache/
    projectn/
        buildout.cfg
        hooks/
        patches/
        parts/
            site-packages-2.4
            site-packages-2.5

django/
     project1/
     ...
     projectn/
zope/
     project1/
     ...
     projectn/

anotherCategory/
    anotherProject/

minilays/
    eggs/
    dependencies/
    instances/
    meta/
    samples/
    anExternalMinilay/
\end{Verbatim}


\subsubsection{Layout explanation}
\label{spec:layout-explanation}\begin{description}
\item[{\emph{bin/minimerge}:}] \leavevmode\begin{itemize}
\item {} 
The project Assembler.

\end{itemize}

\item[{\emph{etc/minimerge.cfg}:}] \leavevmode
Minitage configuration file.

\item[{\emph{dependencies/}:}] \leavevmode\begin{itemize}
\item {} 
Libraries and applications like libpng, python-2.4 or readline.

\item {} 
One dependency per directory.

\item {} 
The installation prefix for each dependency is:

\begin{Verbatim}[commandchars=\\\{\}]
\PYG{n}{dependencies}\PYG{o}{/}\PYG{n}{dependency}\PYG{o}{-}\PYG{n}{name}\PYG{o}{/}\PYG{n}{parts}\PYG{o}{/}\PYG{n}{part}
\end{Verbatim}

\end{itemize}

\item[{\emph{eggs/}:}] \leavevmode
They is two possibilities there:
\begin{itemize}
\item {} 
In a particular eggs/directory:
\begin{itemize}
\item {} 
Traditional distutilized python modules

\item {} 
Python modules shipped is a non pythonish way (like libxml2)

\item {} 
They must install a sub site-packages for each python version supported:

\begin{Verbatim}[commandchars=@\[\]]
eggs/egg/
    site-packages-2.4/
        module/@_@_init@_@_.py
    site-packages-2.5/
        module/@_@_init@_@_.py
    site-packages-2.6/
        module/@_@_init@_@_.py
    site-packages-3.0/
        module/@_@_init@_@_.py
\end{Verbatim}

\end{itemize}

\item {} \begin{description}
\item[{Python eggyfiables modules}] \leavevmode\begin{itemize}
\item {} \begin{description}
\item[{They are installed in the ``eggs-cache''}] \leavevmode\begin{itemize}
\item {} 
eggs in release mode:

\begin{Verbatim}[commandchars=\\\{\}]
\PYG{n}{eggs}\PYG{o}{/}\PYG{n}{cache}
\end{Verbatim}

\end{itemize}

\end{description}

\end{itemize}

\end{description}

\end{itemize}

\item[{django/:}] \leavevmode\begin{itemize}
\item {} 
Django projects.

\end{itemize}

\item[{\emph{zope/}:}] \leavevmode\begin{itemize}
\item {} 
Zope/Plone projects which only install zope, plone and the needed products.

\item {} 
Just think to add the needed site-packages in the project's extra-path so that buildout can find them!

\item {} 
Do not use not packaged eggs parts there or BURN IN HELL!

\end{itemize}

\item[{misc:/}] \leavevmode\begin{itemize}
\item {} 
All that cannot be elsewhere

\end{itemize}

\item[{tg/:}] \leavevmode\begin{itemize}
\item {} 
Turbogears project

\end{itemize}

\item[{\emph{minilays/}}] \leavevmode{[}\emph{dependencies} \textbar{} \emph{zope} \textbar{} \emph{django} \textbar{} \emph{eggs}{]}
Those are MINILAYS. Minilays are similar to gentoo `s OVERLAYS. Or, be reference, to entries in your source.list on Debian/Ubuntu.
They contains minibuilds.
Those are the packages that our package manager deals with.
You can add search Directories which are not in \code{minilays/} by setting the ``MINILAYS'' environment variable.
ex:

\begin{Verbatim}[commandchars=\\\{\}]
\PYG{n+nb}{export }\PYG{n+nv}{MINILAYS}\PYG{o}{=}\PYG{l+s+s2}{"\textasciitilde{}/otherminibuildsdirectory"}
\end{Verbatim}

\end{description}


\chapter{How do i use it}
\label{index:how-do-i-use-it}

\section{Installation}
\label{installation:installation}\label{installation::doc}

\subsection{Variables}
\label{installation:variables}\begin{itemize}
\item {} 
\code{\$prefix}: root of minitage

\item {} 
\code{\$ins}: root of a project inside minitage

\end{itemize}


\subsection{Buildout Configuration}
\label{installation:buildout-configuration}
\textbf{To store all downloaded stuff in the same place}, you ll need to set buildout to do so.
\begin{quote}

\begin{Verbatim}[commandchars=\\\{\}]
mkdir -p \textasciitilde{}/.buildout/downloads
cat \PYG{l+s}{\textless{}\textless{} EOF \textgreater{} \textasciitilde{}/.buildout/default.cfg}
\PYG{l+s}{[buildout]}
\PYG{l+s}{download-directory = \$HOME/.buildout/downloads}
\PYG{l+s}{download-cache = \$HOME/.buildout/downloads}
\PYG{l+s}{EOF}
\end{Verbatim}
\end{quote}


\subsection{System Requirements}
\label{installation:system-requirements}
Minitage requires a complete toolchain to build softwares from the c compiler to the autotools.

if you need something that must be compiled, you must have some requirements installed like :
\begin{itemize}
\item {} 
m4

\item {} 
A c compiler (gcc ?)

\item {} 
pkg-config

\item {} 
autotools
\begin{itemize}
\item {} 
automake

\item {} 
autoconf

\end{itemize}

\item {} 
libtool if your platform supports it

\item {} 
groff

\item {} 
man-db, man utils,  whatever your system core man packages are

\end{itemize}


\subsubsection{Debian/Ubuntu}
\label{installation:debian-ubuntu}\begin{quote}

Prior to begin with the project, ensure those dependencies are installed on your system :
\begin{itemize}
\item {} 
m4

\item {} 
build-essential

\item {} 
pkg-config

\item {} 
automake

\item {} 
libtool

\item {} 
autoconf
\begin{quote}

\begin{Verbatim}[commandchars=\\\{\}]
apt-get install build-essential m4 libtool pkg-config autoconf gettext bzip2 groff man-db automake libsigc++-2.0-dev tcl8.4
or
apt-get install build-essential m4 libtool pkg-config autoconf gettext bzip2 groff man-db automake libsigc++-2.0-dev tcl8.5
\end{Verbatim}
\end{quote}

\end{itemize}
\end{quote}


\subsubsection{FreeBSD}
\label{installation:freebsd}\begin{itemize}
\item {} 
gmake

\item {} 
gsed
\begin{quote}

\begin{Verbatim}[commandchars=\\\{\}]
\PYG{n+nb}{cd} /usr/ports/sysutils/portinstall
make install clean
\PYG{k}{for }i in gsed gmake autotools autoconf;do portinstall \PYG{n+nv}{\$i};\PYG{k}{done}
\end{Verbatim}
\end{quote}

\end{itemize}


\subsubsection{Gentoo}
\label{installation:gentoo}\begin{quote}

Gentoo is perfect by default, no requirement.
\end{quote}


\subsubsection{Suze}
\label{installation:suze}\begin{quote}

Prior to begin with the project, ensure those dependencies are installed on
your system:
\begin{itemize}
\item {} 
toolchain (gcc, autotools, m4)

\end{itemize}
\end{quote}


\subsubsection{Fedora}
\label{installation:fedora}\begin{quote}

Prior to begin with the project, ensure those dependencies are installed on
your system:

This one liner can help you
\begin{quote}

\begin{Verbatim}[commandchars=\\\{\}]
yum install automake autoconf libtool wget gcc-c++ patch
\end{Verbatim}
\end{quote}
\end{quote}


\subsubsection{MacOS X}
\label{installation:macos-x}\begin{quote}

Before starting with the project, be sure that these dependencies are installed via macports (\href{http://macports.org}{http://macports.org}) on your system :
\begin{itemize}
\item {} 
\href{http://trac.macports.org/projects/macports/browser/trunk/dports/archivers/bzip2/Portfile}{bzip2}

\item {} 
\href{http://trac.macports.org/projects/macports/browser/trunk/dports/archivers/gnutar/Portfile}{gnu tar (gtar)}

\item {} 
\href{http://trac.macports.org/projects/macports/browser/trunk/dports/archivers/unzip/Portfile}{unzip}

\item {} 
\href{http://trac.macports.org/projects/macports/browser/trunk/dports/devel/binutils/Portfile}{binutils}

\item {} 
\href{http://trac.macports.org/projects/macports/browser/trunk/dports/textproc/gsed/Portfile}{sed (gsed)}

\item {} 
\href{http://trac.macports.org/projects/macports/browser/trunk/dports/devel/gmake/Portfile}{gnu make (gmake)}

\item {} 
\href{http://trac.macports.org/projects/macports/browser/trunk/dports/devel/autoconf/Portfile}{autoconf}

\item {} 
\href{http://trac.macports.org/projects/macports/browser/trunk/dports/devel/automake/Portfile}{automake}

\item {} 
\href{http://trac.macports.org/projects/macports/browser/trunk/dports/devel/m4/Portfile}{m4}

\end{itemize}

Do not forget to update your bash profile to take your installed ports into account
\begin{quote}

\begin{Verbatim}[commandchars=\\\{\}]
\PYG{c}{\# put this line into \textasciitilde{}/.bashrc and \textasciitilde{}/.bash\PYGZus{}profile}
\PYG{n+nb}{export }\PYG{n+nv}{PATH}\PYG{o}{=}/opt/local/bin:/opt/local/sbin:\PYG{n+nv}{\$PATH}
\end{Verbatim}
\end{quote}
\end{quote}


\subsection{Python}
\label{installation:python}

\subsubsection{Existing python}
\label{installation:existing-python}\begin{quote}

You need a python with distribute, zlib, bz2 and ssl support.

Try that in your interpretery
\begin{quote}

\begin{Verbatim}[commandchars=\\\{\}]
\PYG{g+gp}{\textgreater{}\textgreater{}\textgreater{} }\PYG{k+kn}{import} \PYG{n+nn}{tarfile}
\PYG{g+gp}{\textgreater{}\textgreater{}\textgreater{} }\PYG{k+kn}{import} \PYG{n+nn}{zipfile}
\PYG{g+gp}{\textgreater{}\textgreater{}\textgreater{} }\PYG{k+kn}{import} \PYG{n+nn}{bz2}
\PYG{g+gp}{\textgreater{}\textgreater{}\textgreater{} }\PYG{k+kn}{import} \PYG{n+nn}{\PYGZus{}ssl}
\PYG{g+gp}{\textgreater{}\textgreater{}\textgreater{} }\PYG{k+kn}{import} \PYG{n+nn}{zlib}
\PYG{g+gp}{\textgreater{}\textgreater{}\textgreater{} }\PYG{k+kn}{import} \PYG{n+nn}{setuptools}
\PYG{g+go}{\textgreater{}\textgreater{}\textgreater{}}
\end{Verbatim}
\end{quote}
\end{quote}

Tip:
If you do not want to use a custom compiled python, be sure to have installed python with it's ``dev'' packages and with distribute and virtualenv.
On Debian systems, for example, you can use the following snippet:

\begin{Verbatim}[commandchars=@\[\]]
apt-get install python-dev python
\end{Verbatim}


\subsubsection{PyBootstrapper}
\label{installation:pybootstrapper}

\paragraph{Goal}
\label{installation:goal}
This utility deploys for you a nicely python with all its dependencies in a prefix of your choice.

This script will install for you:
\begin{itemize}
\item {} 
Python-2.5.2 (default) or Python-2.4.5

\item {} 
openssl 0.9.7

\item {} 
zlib-1.2.3

\item {} 
bzip2-1.0.4

\item {} 
ncurses-5.6

\item {} 
readlines-5.2

\item {} 
ez\_setup.py which will install those python packages:
\begin{itemize}
\item {} 
distribute

\item {} 
zc.buildout

\item {} 
PasteScripts

\item {} 
virtualenv

\end{itemize}

\end{itemize}


\paragraph{Variables}
\label{installation:id1}\begin{itemize}
\item {} 
We will use some variables to refer to well known places and scripts.

\item {} 
Just adjust the following code to fit to your needs and type it in your current shell/
\begin{quote}

\begin{Verbatim}[commandchars=\\\{\}]
\PYG{n+nb}{export }\PYG{n+nv}{prefix}\PYG{o}{=}\PYG{n+nv}{\$HOME}/minitage
\PYG{n+nb}{export }\PYG{n+nv}{python}\PYG{o}{=}\PYG{n+nv}{\$HOME}/tools/python
\end{Verbatim}
\end{quote}

\end{itemize}


\paragraph{Usage}
\label{installation:usage}\begin{itemize}
\item {} 
Please use a FULL path with this script!
\begin{quote}

\begin{Verbatim}[commandchars=\\\{\}]
mkdir -p \PYG{n+nv}{\$python}
\PYG{n+nb}{cd} \PYG{n+nv}{\$python}
wget http://git.minitage.org/git/minitage/shell/plain/PyBootstrapper.sh
bash ./PyBootstrapper.sh \PYG{n+nv}{\$python}
\end{Verbatim}
\end{quote}

\end{itemize}


\paragraph{offline mode}
\label{installation:offline-mode}\begin{itemize}
\item {} 
Please use a FULL path with this script!

\item {} 
You can use it in offline mode but put the archives in your \emph{\$python/downloads} eg:
\begin{quote}

\begin{Verbatim}[commandchars=\\\{\}]
ln -s /prod/1.0/downloads  \PYG{n+nv}{\$python}/downloads
bash ./PyBootstrapper.sh  -o \PYG{n+nv}{\$python}
\end{Verbatim}
\end{quote}

\end{itemize}


\subsection{Using virtualenv}
\label{installation:using-virtualenv}
\textbf{You have to use} \href{http://pypi.python.org/pypi/virtualenv}{virtualenv},
minitage fits well with it and requires now \emph{distribute} to run.

virtualenv is a tool that allow you to create isolated Python
environments.
\begin{itemize}
\item {} 
!!!  \emph{PIP} IS NOT SUPPORTED !!!
\begin{quote}

Here is how to set up an environment with it:
\begin{itemize}
\item {} 
\textbf{Only if you do not have used the bootstrap script}, you must install virtualenv:

\begin{Verbatim}[commandchars=@\[\]]
@# maybe sudo ?
wget  http://python-distribute.org/distribute@_setup.py
python distribute@_setup.py
easy@_install -U virtualenv @# remove also any other virtualenv installation
\end{Verbatim}

\item {} 
Install the minitage prefix, this is just a new virtualenv creation:

\begin{Verbatim}[commandchars=@\[\]]
@$python/bin/virtualenv --no-site-packages --distribute @$prefix
@# maybe that if you do not used the bootstrapper
@# virtualenv --no-site-packages --distribute @$prefix
\end{Verbatim}

\item {} 
activate it:

\begin{Verbatim}[commandchars=@\[\]]
source @$prefix/bin/activate
\end{Verbatim}

\end{itemize}

\textbf{KEEP IN MIND THAT YOU MUST ACTIVATE VIRTUALENV AT ANY TIME YOU USE IT.}
\end{quote}

\end{itemize}


\subsection{Installing minitage}
\label{installation:installing-minitage}

\subsubsection{A stable version}
\label{installation:a-stable-version}\begin{quote}

Minitage is a classical python egg, you can get it throught easy\_install (DISTRIBUTE).

To install minitage in a stable version, follow those steps:
\begin{itemize}
\item {} 
!!!  \emph{PIP} IS NOT SUPPORTED !!!

\item {} 
Install minitage
\begin{quote}

\begin{Verbatim}[commandchars=\\\{\}]
\PYG{n+nb}{source} \PYG{n+nv}{\$prefix}/bin/activate
easy\PYGZus{}install -U minitage.core
\end{Verbatim}
\end{quote}

\item {} 
Sync its packages (all its minilays in minitage terminology).

\textbf{This will initiate also all the minitage directories for the first run.}
\begin{quote}

\begin{Verbatim}[commandchars=\\\{\}]
\PYG{n+nb}{source} \PYG{n+nv}{\$prefix}/bin/activate
minimerge -s
\end{Verbatim}
\end{quote}

\end{itemize}
\end{quote}


\subsection{Using minitage}
\label{installation:using-minitage}
Those are usage samples... You have not to run that if you do not need to ;)


\subsubsection{Install python-xxx}
\label{installation:install-python-xxx}\begin{quote}

\begin{Verbatim}[commandchars=\\\{\}]
\PYG{n+nb}{source} \PYG{n+nv}{\$prefix}/bin/activate
minimerge python-xxx
\end{Verbatim}
\end{quote}


\subsubsection{Install a custom minilay}
\label{installation:install-a-custom-minilay}\begin{quote}

\begin{Verbatim}[commandchars=\\\{\}]
\PYG{c}{\# get the project minilay}
\PYG{c}{\# minitage is aware of the MINILAYS environnment variable, you can use it to specify space separated minlays}
scm CHECKOUT  https://subversion.foo.net/YOURPROJECT/minilay/trunk \PYG{n+nv}{\$prefix}/minilays/YOURPROJECTMINILAY
\end{Verbatim}
\end{quote}


\subsubsection{Deploy a project with minitage}
\label{installation:deploy-a-project-with-minitage}\begin{quote}

\begin{Verbatim}[commandchars=\\\{\}]
\PYG{c}{\# get the project minilay}
\PYG{c}{\# minitage is aware of the MINILAYS environnment variable, you can use it to specify space separated minlays}
scm CHECKOUT  https://subversion.foo.net/YOURPROJECT/minilay/trunk \PYG{n+nv}{\$prefix}/minilays/YOURPROJECTMINILAY
\PYG{c}{\# minimerging it}
\PYG{n+nb}{source} \PYG{n+nv}{\$prefix}/bin/activate
minimerge project
\end{Verbatim}
\end{quote}


\subsection{Extra options and usage}
\label{installation:extra-options-and-usage}\begin{quote}

\begin{Verbatim}[commandchars=\\\{\}]
\PYG{n+nb}{source} \PYG{n+nv}{\$prefix}/bin/activate
minimerge  --help
\end{Verbatim}
\end{quote}


\section{Minitage in the daily use of a developer}
\label{usecases/index::doc}\label{usecases/index:minitage-in-the-daily-use-of-a-developer}
Here are some examples and documentation to help you discover the wonderful world of minitage.


\subsection{The minitage swiss knife, \textbf{minimerge}}
\label{usecases/using_minimerge:the-minitage-swiss-knife-minimerge}\label{usecases/using_minimerge::doc}

\subsubsection{Verbose mode}
\label{usecases/using_minimerge:verbose-mode}
Usable with any other option, recommended to use which each command !

\begin{Verbatim}[commandchars=\\\{\}]
\PYG{n}{minimerge} \PYG{o}{-}\PYG{n}{v} \PYG{p}{[}\PYG{n}{opts}\PYG{p}{]} \PYG{p}{[}\PYG{n}{args}\PYG{p}{]}
\end{Verbatim}


\subsubsection{Dry run}
\label{usecases/using_minimerge:dry-run}
\begin{Verbatim}[commandchars=\\\{\}]
\PYG{n}{minimerge} \PYG{p}{[}\PYG{n}{opts}\PYG{p}{]} \PYG{o}{-}\PYG{n}{pv} \PYG{p}{[}\PYG{n}{args}\PYG{p}{]}
\end{Verbatim}


\subsubsection{Ask mode}
\label{usecases/using_minimerge:ask-mode}
minimerge will wait for your confirmation before proceeding to anything:

\begin{Verbatim}[commandchars=\\\{\}]
\PYG{n}{minimerge} \PYG{p}{[}\PYG{n}{opts}\PYG{p}{]} \PYG{o}{-}\PYG{n}{a} \PYG{p}{[}\PYG{n}{args}\PYG{p}{]}
\end{Verbatim}


\subsubsection{Squizzes dependencies}
\label{usecases/using_minimerge:squizzes-dependencies}
\begin{Verbatim}[commandchars=@\[\]]
minimerge -j dep package
\end{Verbatim}

If you want to install \code{a} and \code{a} depends on \code{b}, \code{c}, \code{d} respectivly.
To squizze \code{b} and \code{c} and only install \code{d} and \code{a}:

\begin{Verbatim}[commandchars=@\[\]]
minimerge -j d a
\end{Verbatim}


\subsubsection{Build only dependencies}
\label{usecases/using_minimerge:build-only-dependencies}
\begin{Verbatim}[commandchars=@\[\]]
minimerge --only-dependencies package
\end{Verbatim}


\subsubsection{Update your minimlays}
\label{usecases/using_minimerge:update-your-minimlays}
\begin{Verbatim}[commandchars=\\\{\}]
\PYG{n}{minimerge} \PYG{o}{-}\PYG{n}{s}
\end{Verbatim}


\subsubsection{Install a package and its dependencies}
\label{usecases/using_minimerge:install-a-package-and-its-dependencies}
\begin{Verbatim}[commandchars=@\[\]]
minimerge package
\end{Verbatim}


\subsubsection{Install a package without dealing with dependencies}
\label{usecases/using_minimerge:install-a-package-without-dealing-with-dependencies}
That will run only your package installation:

\begin{Verbatim}[commandchars=@\[\]]
minimerge -N package
\end{Verbatim}


\subsubsection{Update the package codebase}
\label{usecases/using_minimerge:update-the-package-codebase}
That will run the update method of what you have used to checkout your
package:

\begin{Verbatim}[commandchars=@\[\]]
minimerge -U package
\end{Verbatim}


\subsubsection{Run again the installation of a package}
\label{usecases/using_minimerge:run-again-the-installation-of-a-package}
That will run the update method of what you have used to checkout your
package:

\begin{Verbatim}[commandchars=@\[\]]
minimerge -u package
\end{Verbatim}


\subsubsection{Unconditionnally install or reinstall a package}
\label{usecases/using_minimerge:unconditionnally-install-or-reinstall-a-package}
\begin{Verbatim}[commandchars=@\[\]]
minimerge -R package
\end{Verbatim}


\subsubsection{Note}
\label{usecases/using_minimerge:note}
All the previous options can be combined.


\subsubsection{Update/Rebuid  a package and all its dependencies}
\label{usecases/using_minimerge:update-rebuid-a-package-and-all-its-dependencies}
To install \& rebuild a package and all its dependencies:

\begin{Verbatim}[commandchars=@\[\]]
minimerge -s
minimerge -Uu package
\end{Verbatim}


\subsubsection{Behaviour notes}
\label{usecases/using_minimerge:behaviour-notes}\begin{itemize}
\item {} 
Minitage will never run automaticly again where a buildout has been
already run and has left an `installed.cfg' file. You must tell it
explicitly to do that like:
\begin{itemize}
\item {} 
Just run the buildout:

\begin{Verbatim}[commandchars=@\[\]]
minimerge -NUu package
or
minimerge -Nu package
\end{Verbatim}

\item {} 
Remove the .installed.cfg and run buildout:

\begin{Verbatim}[commandchars=@\[\]]
minimerge  -NRU package
minimerge  -NR package
\end{Verbatim}

\end{itemize}

\end{itemize}


\subsubsection{Relaunch an interrupted minitage}
\label{usecases/using_minimerge:relaunch-an-interrupted-minitage}
\begin{Verbatim}[commandchars=@\[\]]
minimerge -u mypackage
\end{Verbatim}


\subsubsection{Relaunch a failed installation exactly where it stopped}
\label{usecases/using_minimerge:relaunch-a-failed-installation-exactly-where-it-stopped}
\begin{Verbatim}[commandchars=@\[\]]
minimerge -uj myfailedpackage package
\end{Verbatim}


\subsection{Beginning a project with minitage: The Initial Steps}
\label{usecases/begin_project:beginning-a-project-with-minitage-the-initial-steps}\label{usecases/begin_project::doc}

\subsubsection{Project initialization}
\label{usecases/begin_project:project-initialization}\begin{itemize}
\item {} 
Choose an existing  {\hyperref[paster/index:minitageprojects]{\emph{minitage layout}}} if any suits your needs.

\item {} 
Create the project itself and package it to be `minitage compliant'
(uploading, versionining). To quickstart, just just an existing minitage template.

\item {} 
Create or edit a minibuild, in a minilays  which points to this layout:

\begin{Verbatim}[commandchars=@\[\]]
mkdir -p @$mt/minilays/mynewminilay
vim  @$mt/minilays/mynewminilay/project
\end{Verbatim}

\item {} 
Maybe, it is a good time to version the minilay.

\item {} 
Lets go for the minimerge dance:

\begin{Verbatim}[commandchars=@\[\]]
minimerge myproject
\end{Verbatim}

\end{itemize}


\subsubsection{Project versionning}
\label{usecases/begin_project:versioning-project}\label{usecases/begin_project:project-versionning}\begin{itemize}
\item {} 
Play around and when your are ready, just version in this way:
\begin{itemize}
\item {} 
\$mt/category/project      -\textgreater{} \href{http://url/buildouts/\$category/myproject}{http://url/buildouts/\$category/myproject}

\item {} 
\$mt/minilays/yourproject  -\textgreater{} \href{http://url/minilays/myproject}{http://url/minilays/myproject}

\end{itemize}

\item {} 
Don't forget to adapt the \emph{src\_uri} or your minibuild:

\begin{Verbatim}[commandchars=@\[\]]
@PYGZlb[]minibuild@PYGZrb[]
...
src@_uri=http://url/buildouts/@$category/myproject
...
\end{Verbatim}

\end{itemize}


\subsection{Porting an exising buildout based project to be a good minitage citizen}
\label{usecases/maintain_project:porting-an-exising-buildout-based-project-to-be-a-good-minitage-citizen}\label{usecases/maintain_project::doc}
Althought you are not obliged to use minitage features, we would recommend to follow the next steps
\begin{itemize}
\item {} 
make a python section pointing to the minitage python installation choosen:

\begin{Verbatim}[commandchars=@\[\]]
@PYGZlb[]python@PYGZrb[]
executable = @${buildout:directory}/../../dependencies/python-X.XX/parts/part/bin/python

@PYGZlb[]buildout@PYGZrb[]
python=python
\end{Verbatim}

\item {} 
Use the common egg cache to drop your built eggs:

\begin{Verbatim}[commandchars=@\[\]]
@PYGZlb[]buildout@PYGZrb[]
eggs-directory=@${buildout:directory}/../../eggs/cache
\end{Verbatim}

\item {} 
Replace recipes with minitage ones if the equivalent exists
\begin{itemize}
\item {} 
\code{zc.recipe.egg}, \code{zc.recipe.egg:scripts} -\textgreater{} \code{minitage.recipe:scripts}

\item {} 
\code{something:cmmi} -\textgreater{} \code{minitage.recipe.cmmi}

\end{itemize}

\item {} 
Read \href{http://pypi.python.org/pypi/minitage.recipe}{http://pypi.python.org/pypi/minitage.recipe} to understand what do the recipes

\item {} 
Make a minibuild pointing to your project:

\begin{Verbatim}[commandchars=@\[\]]
@PYGZlb[]minibuild@PYGZrb[]
dependencies=openssl-0.9 libxml2-2.6 libxslt-1.1 zlib-1.2  py-libxml2-2.6 py-libxslt-1.1 python-2.4 ...
install@_method=buildout
src@_uri=http://hg.foo.net
src@_type=hg
category=zope
homepage=http://foo.net
description= a plone 3.1 buildout for plony
\end{Verbatim}

\item {} 
Version the whole and letz go. (see {\hyperref[usecases/begin_project:versioning-project]{\emph{Project versionning}}}).

\end{itemize}


\subsection{Installation with a python 2.5}
\label{usecases/install25:installation-with-a-python-2-5}\label{usecases/install25::doc}
First some variables:

\begin{Verbatim}[commandchars=@\[\]]
export prefix=/somewhere
export python=@$prefix/python
export minitage=@$prefix/minitage
export ins=@$minitage/zope/myproject
\end{Verbatim}

Bootstrap python:

\begin{Verbatim}[commandchars=@\[\]]
mkdir -p @$python
cd @$python
wget http://git.minitage.org/git/minitage/shell/plain/PyBootstrapper.sh
bash ./PyBootstrapper.sh @$python
\end{Verbatim}

Make a minitage instance:

\begin{Verbatim}[commandchars=@\[\]]
@$python/bin/virtualenv --no-site-packages @$minitage
source @$minitage/bin/activate
easy@_install -U minitage.core minitage.paste
minimerge -s
\end{Verbatim}

Install python-2.5:

\begin{Verbatim}[commandchars=@\[\]]
minimerge python-2.5
\end{Verbatim}


\subsection{A plone3 project with relstorage (postgresql) and varnish2 support}
\label{usecases/plone3:a-plone3-project-with-relstorage-postgresql-and-varnish2-support}\label{usecases/plone3::doc}
First some variables:

\begin{Verbatim}[commandchars=@\[\]]
export prefix=/somewhere
export python=@$prefix/python
export minitage=@$prefix/minitage
export ins=@$minitage/zope/plony
export paster=@$mt/bin/paster
\end{Verbatim}

Bootstrap python:

\begin{Verbatim}[commandchars=@\[\]]
mkdir -p @$python
cd @$python
wget http://git.minitage.org/git/minitage/shell/plain/PyBootstrapper.sh
bash ./PyBootstrapper.sh @$python
\end{Verbatim}

Make a minitage instance:

\begin{Verbatim}[commandchars=@\[\]]
@$python/bin/virtualenv --no-site-packages @$minitage
source @$minitage/bin/activate
easy@_install -U minitage.core minitage.paste
minimerge -s
\end{Verbatim}

Create and install a \textbf{plone3 with RelsSorage project} with the \textbf{varnish2} and \textbf{postgresql} profiles on top of it:

\begin{Verbatim}[commandchars=@\[\]]
@$paster create -t minitage.plone3 plony with@_psycopg2=yes mode=relstorage
minimerge plony
@$paster create -t minitage.instances.varnish2 plony
@$paster create -t minitage.instances.postgresql plony
\end{Verbatim}

Launch the varnish if you want (you may edit the VCL in \$ins/sys/etc/varnish before):

\begin{Verbatim}[commandchars=@\[\]]
cd @$mt/zope/plony
@$ins/sys/etc/init.d/plony-varnish2 restart
@$ins/sys/etc/init.d/postgresql@_plony.plony restart
@$ins/bin/instance fg
\end{Verbatim}

You can verify that you have things in the db:

\begin{Verbatim}[commandchars=@\[\]]
@$sys/bin/plony.psql
\end{Verbatim}

Play around and when your are ready, just version in this way:
\begin{itemize}
\item {} 
\$mt/category/project      -\textgreater{} \href{http://url/buildouts/\$category/plony}{http://url/buildouts/\$category/plony}

\item {} 
\$mt/minilays/yourproject  -\textgreater{} \href{http://url/minilays/plony}{http://url/minilays/plony}

\end{itemize}

Of course, here category is zope.

Don't forget to adapt the \emph{src\_uri} or your minibuild:

\begin{Verbatim}[commandchars=@\[\]]
@PYGZlb[]minibuild@PYGZrb[]
...
src@_uri=http://url/buildouts/@$category/plony
...
\end{Verbatim}


\subsection{Install a project with a CAS server wired on an OpenLDAP backend}
\label{usecases/deploying_a_cas_server::doc}\label{usecases/deploying_a_cas_server:install-a-project-with-a-cas-server-wired-on-an-openldap-backend}

\subsubsection{Minitage}
\label{usecases/deploying_a_cas_server:minitage}
Installer minitage as the official documentation says. \code{http://www.minitage.org/installation.html}.

Take \$MT as the minitage installation prefix.

\begin{Verbatim}[commandchars=@\[\]]
MT=@$HOME/minitage/
\end{Verbatim}

Be up to date before each minitage commande usage

\begin{Verbatim}[commandchars=@\[\]]
easy@_install -U minitage.core minitage.paste minitage.paste.extras
\end{Verbatim}

Right after minitage installation, issue the following commands

\begin{Verbatim}[commandchars=@\[\]]
cd @$MT/minilays
svn co https://yoururl/yourminilay
cd @$MT
minimerge myproject
paster create -t minitage.instances.env myproject
. @$MT/zope/myproject/sys/share/minitage/minitage.env
\end{Verbatim}

Now, there are 3 tasks left:
\begin{itemize}
\item {} 
LDAP server configuration

\item {} 
CAS (sso) server configuration

\end{itemize}


\subsubsection{Ldap configuration}
\label{usecases/deploying_a_cas_server:ldap-configuration}
Install a non privilegied slapd instance.

\begin{Verbatim}[commandchars=@\[\]]
passwd=@$(whoami)
@$MT/bin/paster create --no-interactive -t minitage.instances.openldap myproject db@_orga=my-domain db@_suffix=com db@_port=1389 ssl@_port=1636
cd @$INS/sys/etc/openldap
\end{Verbatim}

Include the right schemas in the slapd configuration file:

\begin{Verbatim}[commandchars=@\[\]]
- core
- cosine
- inetorgperson
- nis:
\end{Verbatim}

Comment out the additional schemas.

\begin{Verbatim}[commandchars=@\[\]]
@$EDITOR @$INS/etc/openldap/myproject@_my-domain.com-slapd.conf
\end{Verbatim}

Create the base ldif

\begin{Verbatim}[commandchars=@\[\]]
cat @textless[]@textless[] EOF@textbar[]sed "s/    //g" @textgreater[] /tmp/initial.ldif
dn: dc=my-domain,dc=com
objectClass: dcObject
objectClass: organization
o: Example Company
dc: my-domain
structuralObjectClass: organization
entryUUID: bde2ec54-fc08-102d-9c3e-0d9dd31353b4
creatorsName: cn=Manager,dc=my-domain,dc=com
createTimestamp: 20090703103350Z
entryCSN: 20090703103350.268281Z@#000000@#000@#000000
modifiersName: cn=Manager,dc=my-domain,dc=com
modifyTimestamp: 20090703103350Z

dn: cn=Manager,dc=my-domain,dc=com
objectClass: organizationalRole
cn: Manager
structuralObjectClass: organizationalRole
entryUUID: be07d7bc-fc08-102d-9c3f-0d9dd31353b4
creatorsName: cn=Manager,dc=my-domain,dc=com
createTimestamp: 20090703103350Z
entryCSN: 20090703103350.510240Z@#000000@#000@#000000
modifiersName: cn=Manager,dc=my-domain,dc=com
modifyTimestamp: 20090703103350Z

dn: cn=Professionnels,dc=my-domain,dc=com
objectClass: top
objectClass: groupOfUniqueNames
uniqueMember: cn=Manager,dc=my-domain,dc=com
uniqueMember: cn=hpotter,dc=my-domain,dc=com
cn: Professionnels
structuralObjectClass: groupOfUniqueNames
entryUUID: f5e1d890-fc08-102d-9c40-0d9dd31353b4
creatorsName: cn=Manager,dc=my-domain,dc=com
createTimestamp: 20090703103524Z
entryCSN: 20090703142607.008076Z@#000000@#000@#000000
modifiersName: cn=Manager,dc=my-domain,dc=com
modifyTimestamp: 20090703142607Z

dn: ou=People,dc=my-domain,dc=com
ou: People
objectClass: top
objectClass: organizationalUnit
objectClass: domainRelatedObject
associatedDomain: my-domain.com

EOF
\end{Verbatim}

Load the base ldif

\begin{Verbatim}[commandchars=@\[\]]
source @$INS/sys/share/minitage/mintiage.env
@$INS/sys//etc/init.d/openldap@_myproject@_my-domain.com stop
my-domain.com.slapadd  -l /tmp/initial.ldif
@$INS/sys//etc/init.d/openldap@_myproject@_my-domain.com start
\end{Verbatim}

Check

\begin{Verbatim}[commandchars=@\[\]]
my-domain.com.ldapsearch -w@$(whoami)
\end{Verbatim}

How to change the password

\begin{Verbatim}[commandchars=@\[\]]
passwd=@$(my-domain.com.slappasswd  -h'{MD5}');sed -re "s/\{MD.*/@$passwd/g" -i @$INS/sys/etc/openldap/myproject@_my-domain.com-slapd.conf
@$INS/sys//etc/init.d/openldap@_myproject@_my-domain.com restart
\end{Verbatim}

Check

\begin{Verbatim}[commandchars=@\[\]]
my-domain.com.ldapsearch -w@$passwd
\end{Verbatim}

At this stage, your OpenLDAP server must be configured and listening for input connections.


\subsubsection{CAS Server configuration}
\label{usecases/deploying_a_cas_server:cas-server-configuration}
First, you must have a JDK configured and integrated (JAVA\_HOME) under the hood
For thus to be done, on Debian/Ubuntu, you can issue the followiung commands

\begin{Verbatim}[commandchars=@\[\]]
cd @$MT
source bin/activate
sudo apt-get install openjdk-6-jdk
export JAVA@_HOME=/usr/lib/jvm/java-6-openjdk/
\end{Verbatim}

Assure to have java in your path and JAVA\_HOME defined unless tomcat will not install

Install tomcat

\begin{Verbatim}[commandchars=@\[\]]
@# add tomcat-6.0.20 to minilays/yourminilay/myproject
vim @$MT/minilays/yourminilay/myproject
minimerge -v myproject
\end{Verbatim}

Install a CAS server instance

\begin{Verbatim}[commandchars=@\[\]]
@$MT/bin/paster create -t minitage.instances.cas --no-interactive myproject http@_port=8000
\end{Verbatim}

Notes:
\begin{itemize}
\item {} 
Default admin credentials:  \$(whoami)/secret

\item {} 
Please have a look on the generated README in \$INS

\end{itemize}

Configure the LDAP server as an authentication backend for the CAS server (\href{http://www.ja-sig.org/wiki/display/CASUM/LDAP}{http://www.ja-sig.org/wiki/display/CASUM/LDAP}):
\begin{itemize}
\item {} 
comment the bean \emph{org.jasig.cas.authentication.handler.support.SimpleTestUsernamePasswordAuthenticationHandler} around the li.91 in \$INS/sys/var/data/tomcat/cas/webapps/cas/WEB-INF/deployerConfigContext.xml:

\begin{Verbatim}[commandchars=@\[\]]
@textless[]!--    @textless[]bean class="org.jasig.cas.authentication.handler.support.SimpleTestUsernamePasswordAuthenticationHandler" /@textgreater[]--@textgreater[]
\end{Verbatim}

\item {} 
add the following Lines under the commented line and adapt the urls:

\begin{Verbatim}[commandchars=@\[\]]
@textless[]bean
    class="org.jasig.cas.adaptors.ldap.FastBindLdapAuthenticationHandler"@textgreater[]
    @textless[]property name="filter" value="uid=@%u,ou=People,dc=my-domain,dc=com" /@textgreater[]
    @textless[]property name="contextSource" ref="contextSource" /@textgreater[]
@textless[]/bean@textgreater[]
\end{Verbatim}

\item {} 
under the bean containing the previous one, just add:

\begin{Verbatim}[commandchars=@\[\]]
    @textless[]bean id="contextSource" class="org.springframework.ldap.core.support.LdapContextSource"@textgreater[]
            @textless[]property name="anonymousReadOnly" value="false"/@textgreater[]
            @textless[]property name="pooled" value="true"/@textgreater[]
            @textless[]property name="urls"@textgreater[]
                    @textless[]list@textgreater[]
        @textless[]!-- TO ENABLE SSL,  YOU MUST INSTALL THE SERVER CERTIFICATE INE THE JAVA SSL KEYSTORE
                            @textless[]value@textgreater[]ldaps://localhost:1389/@textless[]/value@textgreater[]
        --@textgreater[]
                            @textless[]value@textgreater[]ldap://localhost:1389/@textless[]/value@textgreater[]
                    @textless[]/list@textgreater[]
            @textless[]/property@textgreater[]
            @textless[]property name="baseEnvironmentProperties"@textgreater[]
                    @textless[]map@textgreater[]
        @textless[]!--
                            @textless[]entry@textgreater[]
                                    @textless[]key@textgreater[]
                                            @textless[]value@textgreater[]java.naming.security.protocol@textless[]/value@textgreater[]
                                    @textless[]/key@textgreater[]
                                    @textless[]value@textgreater[]ssl@textless[]/value@textgreater[]
                            @textless[]/entry@textgreater[]
        --@textgreater[]
                            @textless[]entry@textgreater[]
                                    @textless[]key@textgreater[]
                                            @textless[]value@textgreater[]java.naming.security.authentication@textless[]/value@textgreater[]
                                    @textless[]/key@textgreater[]
                                    @textless[]value@textgreater[]simple@textless[]/value@textgreater[]
                            @textless[]/entry@textgreater[]
                    @textless[]/map@textgreater[]
            @textless[]/property@textgreater[]
    @textless[]/bean@textgreater[]

- Create some test users ::

            cat @textless[]@textless[] EOF@textbar[]sed "s/                    //g" @textgreater[] /tmp/test.ldif
            dn: uid=test1,ou=People,dc=my-domain,dc=com
            uid: test1
            cn: test1
            sn: test1
            mail: test1@PYGZat[]my-domain.com
            objectClass: person
            objectClass: organizationalPerson
            objectClass: inetOrgPerson
            objectClass: top
            userPassword: foo

            dn: uid=test,ou=People,dc=my-domain,dc=com
            uid: test
            cn: test
            sn: test
            mail: test@PYGZat[]my-domain.com
            objectClass: person
            objectClass: organizationalPerson
            objectClass: inetOrgPerson
            objectClass: top
            userPassword: foo
            EOF
\end{Verbatim}

\end{itemize}

Insert:

\begin{Verbatim}[commandchars=\\\{\}]
\PYG{n}{my}\PYG{o}{-}\PYG{n}{domain}\PYG{o}{.}\PYG{n}{com}\PYG{o}{.}\PYG{n}{ldapadd} \PYG{o}{-}\PYG{n}{f} \PYG{o}{/}\PYG{n}{tmp}\PYG{o}{/}\PYG{n}{test}\PYG{o}{.}\PYG{n}{ldif}
\end{Verbatim}


\subsubsection{Checks}
\label{usecases/deploying_a_cas_server:checks}
I can connect; see my own password but not the other one's

\begin{Verbatim}[commandchars=@\[\]]
@$ my-domain.com.ldapsearch -D uid=test,ou=People,dc=my-domain,dc=com -x -wfoo uid=test1
@# extended LDIF
@#
@# LDAPv3
@# base @textless[]dc=my-domain,dc=com@textgreater[] (default) with scope subtree
@# filter: uid=test1
@# requesting: ALL
@#

@# test1, People, my-domain.com
dn: uid=test1,ou=People,dc=my-domain,dc=com
uid: test1
cn: test1
sn: test1
mail: test1@PYGZat[]my-domain.com
objectClass: person
objectClass: organizationalPerson
objectClass: inetOrgPerson
objectClass: top

@# search result
search: 2
result: 0 Success

@# numResponses: 2
@# numEntries: 1
@$ my-domain.com.ldapsearch -D uid=test,ou=People,dc=my-domain,dc=com -x -wfoo uid=test
@# extended LDIF
@#
@# LDAPv3
@# base @textless[]dc=my-domain,dc=com@textgreater[] (default) with scope subtree
@# filter: uid=test
@# requesting: ALL
@#

@# test, People, my-domain.com
dn: uid=test,ou=People,dc=my-domain,dc=com
uid: test
cn: test
sn: test
mail: test@PYGZat[]my-domain.com
objectClass: person
objectClass: organizationalPerson
objectClass: inetOrgPerson
objectClass: top
userPassword:: Zm9v

@# search result
search: 2
result: 0 Success

@# numResponses: 2
@# numEntries: 1
\end{Verbatim}

I can authenticate on the CAS server (TTW) with the test1/foo login

\begin{Verbatim}[commandchars=@\[\]]
Connexion réussie

Vous vous êtes authentifié(e) auprès du Service Central d'Authentification.

Pour des raisons de sécurité, veuillez vous déconnecter et fermer votre navigateur lorsque vous avez fini d'accéder aux services authentifiés.
\end{Verbatim}


\subsection{FAQ}
\label{usecases/FAQ:faq}\label{usecases/FAQ::doc}\begin{itemize}
\item {} 
Can i tell to minitage to use  something else that \code{buildout.cfg} ?
Yes, it is possible:

\begin{Verbatim}[commandchars=\\\{\}]
\PYG{p}{[}\PYG{n}{minibuild}\PYG{p}{]}
\PYG{o}{.}\PYG{o}{.}\PYG{o}{.}
\PYG{n}{buildout\PYGZus{}config}\PYG{o}{=}\PYG{n}{dev}\PYG{o}{.}\PYG{n}{cfg}
\end{Verbatim}

\end{itemize}


\section{Minitage and paster templates}
\label{paster/index:minitage-and-paster-templates}\label{paster/index::doc}\label{paster/index:minitageprojects}\begin{itemize}
\item {} 
minitage relies activelly on paster templates to quickstart rapidly a new project or a new application instance on an environement(/and project).

\item {} 
As we were very happy with thoses templates, we choosed to make them available even if you are not inside a minitage environment.
Despite, it will lack the minitage integration of getting dependencies in the environement, you will have the possibility to
generate the needed template, somewhere.

\item {} 
There are many kind of templates you can use right now, and an infinty of possibilities to extend them.
Just take your time to appreciate the next sections which will present you in details the available templates.

\end{itemize}


\subsection{Minitage and projects templates}
\label{paster/projects/index::doc}\label{paster/projects/index:minitage-and-projects-templates}

\subsubsection{paster is your friend}
\label{paster/projects/index:paster-is-your-friend}\begin{itemize}
\item {} 
paster from  \href{http://pypi.python.org/pypi/PasteScript/}{PasteScripts} will assist you in your project creation.
If you do not have installed minitage.paste (PasteScript is a dependency,
it would be installed when you install minitage.paste), do it now:

\begin{Verbatim}[commandchars=@\[\]]
@$mt/bin/easy@_install -U minitage.paste
\end{Verbatim}

\end{itemize}
\begin{itemize}
\item {} 
Just answer to a bunch of question and it will generate a layout saving you from a lot of edit time.

\item {} 
minitage is optionnal, you can use the templates without.

\end{itemize}

Basicly it will create the following things:
\begin{itemize}
\item {} 
A project layout

\item {} 
If you choosed to answer yes to \code{"are you inside a minitage"}, a minilay containing:
\begin{itemize}
\item {} 
minibuild(s) (inside the previous minilay) which points to the project layout.

\end{itemize}

\end{itemize}


\subsubsection{Listing the available templates}
\label{paster/projects/index:listing-the-available-templates}
\begin{Verbatim}[commandchars=@\[\]]
@$mt/bin/paster create --list-templates @textbar[] grep minitage @textbar[] grep -v instances
\end{Verbatim}


\subsubsection{Generating a minitage project from a paster template}
\label{paster/projects/index:generating-a-minitage-project-from-a-paster-template}
The projects are just specialized paster templates so you just have to apply a template to a project.

There are two ways to use one project, insde or outside minitage, just by answering  \textbf{yes} or \textbf{no} to the minitage presence question.

For example, we will use the `minitage.plone3' project to generate a base for \textbf{`myproject'}
\begin{itemize}
\item {} 
inside minitage:

\begin{Verbatim}[commandchars=@\[\]]
@$mt/bin/paster create -t minitage.plone3 myproject inside@_minitage=yes
@# then look in @$mt/categ/project and @$mt/minilays/project
\end{Verbatim}

\item {} 
without minitage:

\begin{Verbatim}[commandchars=@\[\]]
paster create -t minitage.plone3 myproject inside@_minitage=no
@# then look in ./project
\end{Verbatim}

\end{itemize}


\subsection{Available projects}
\label{paster/projects/index:available-projects}

\subsubsection{The minitage buildout recipes}
\label{paster/projects/packaging_projects:the-minitage-buildout-recipes}\label{paster/projects/packaging_projects:depproject}\label{paster/projects/packaging_projects::doc}
You must look on : \href{http://pypi.python.org/pypi/minitage.recipe}{http://pypi.python.org/pypi/minitage.recipe} to enjoy playing
with a minitage environment, those recipes fit well inside a minitage.


\subsubsection{Dependency project}
\label{paster/projects/packaging_projects:dependency-project}

\paragraph{Purpose}
\label{paster/projects/packaging_projects:purpose}
How to package a system dependency which is not python related software (not installed with distutils or setuptools) in minitage.
The minitage category is ``dependencies''.


\paragraph{Layout}
\label{paster/projects/packaging_projects:layout}
You must install into minitage/dependencies/dependencyName/parts/part


\paragraph{Sample Layout}
\label{paster/projects/packaging_projects:sample-layout}
You can use this sample throught paster:

\begin{Verbatim}[commandchars=@\[\]]
easy@_install minitage.paste
paster create -t minitage.dependency dependencyName
\end{Verbatim}


\subsubsection{Python based projet}
\label{paster/projects/packaging_projects:eggproject}\label{paster/projects/packaging_projects:python-based-projet}

\paragraph{Purpose}
\label{paster/projects/packaging_projects:id1}
Packaging python things, distutils or setuptools.
The minitage category is ``eggs''.


\paragraph{Distutils}
\label{paster/projects/packaging_projects:distutils}

\subparagraph{Layout}
\label{paster/projects/packaging_projects:id2}
For each python version supported, we will install a site-package-PYTHONVER in the parts buildout directory.


\subparagraph{The minitage.recipe:du recipe}
\label{paster/projects/packaging_projects:the-minitage-recipe-du-recipe}
This recipe can install packages based on distutils that dont support yet setuptools.
In this case, the goal is to install in a prefix the python module to make it
come later in the python patn.

You can create a project based on this recipe with:

\begin{Verbatim}[commandchars=@\[\]]
easy@_install minitage.paste
paster create -t minitage.distutils myproject
\end{Verbatim}


\paragraph{Python eggs}
\label{paster/projects/packaging_projects:python-eggs}

\subparagraph{Layout}
\label{paster/projects/packaging_projects:id3}
The resulted eggs will be put in the egg cache of the buildout section, those projects live in the eggs/ directory


\subparagraph{The minitage.recipe:egg recipe}
\label{paster/projects/packaging_projects:the-minitage-recipe-egg-recipe}
This recipe can install packages based on setuptools.

You can either install from an url or an egg name.

You can install multiple eggs at a time.

You can create a project based on this recipe with:

\begin{Verbatim}[commandchars=@\[\]]
easy@_install minitage.paste
paster create -t minitage.egg myproject
\end{Verbatim}
\phantomsection\label{paster/projects/model_projects:minitagezope}

\subsubsection{Plone 2.5}
\label{paster/projects/model_projects:minitagezope}\label{paster/projects/model_projects:minitageplone25}\label{paster/projects/model_projects:plone-2-5}\label{paster/projects/model_projects::doc}\begin{itemize}
\item {} 
template : \textbf{minitage.plone25}

\item {} 
minitage category : \textbf{zope}

\item {} 
This template is simple and just there for posterity, Plone 25 is old and deprecated !

\item {} 
template initialization and project start:

\begin{Verbatim}[commandchars=@\[\]]
paster create -t minitage.plone25 myproject
cd /minitage/zope/myproject/ @&@& bin/instance restart
\end{Verbatim}

\end{itemize}


\subsubsection{Zope2}
\label{paster/projects/model_projects:zope2}\label{paster/projects/model_projects:minitagezope2}\begin{itemize}
\item {} 
template : \textbf{minitage.zope2}

\item {} 
minitage category : \textbf{zope}

\item {} 
template initialization and starting your project:

\begin{Verbatim}[commandchars=@\[\]]
paster create -t minitage.zope2 myproject
cd /minitage/zope/myproject/ @&@& bin/instance restart
\end{Verbatim}

\end{itemize}


\subsubsection{Plone 3}
\label{paster/projects/model_projects:plone-3}\label{paster/projects/model_projects:minitageplone3}\begin{itemize}
\item {} 
template : \textbf{minitage.plone3}

\item {} 
minitage category : \textbf{zope}

\item {} 
Features:
\begin{itemize}
\item {} 
A buildout for development mode with additionnal tools for debugging your application, IDE integration and DEBUG mode enabled.

\item {} 
A buildout for production

\item {} 
Paster configurations files to launch your Plone using WSGI

\item {} 
Support forZODB, RelStorage, ZEO

\item {} 
You can configure on the fly eggs, zcml and additionnal products

\end{itemize}

\item {} 
Layout additional infos:
\begin{itemize}
\item {} 
/minilay/project/:

\begin{Verbatim}[commandchars=\\\{\}]
\PYG{n}{project}\PYG{p}{,} \PYG{n}{project}\PYG{o}{-}\PYG{n}{dev} \PYG{c}{\# minibuilds for installing your project in production or develop mode}
\end{Verbatim}

\item {} 
/category/project/etc/:

\begin{Verbatim}[commandchars=\\\{\}]
\PYG{p}{[}\PYG{n}{dev}\PYG{o}{\textbar{}}\PYG{n}{prod}\PYG{p}{]}\PYG{o}{.}\PYG{n}{ini} \PYG{c}{\# Paste configuration files}
\PYG{n}{relstorage}\PYG{o}{.}\PYG{n}{cfg}\PYG{p}{,} \PYG{n}{zeo}\PYG{o}{.}\PYG{n}{cfg}\PYG{p}{,} \PYG{n}{zodb}\PYG{o}{.}\PYG{n}{cfg} \PYG{c}{\# buildout specific configuration files}
\PYG{n}{versions}\PYG{o}{.}\PYG{n}{cfg} \PYG{c}{\# Plone KGS}
\end{Verbatim}

\item {} 
/category/project/:

\begin{Verbatim}[commandchars=\\\{\}]
\PYG{n}{buildout}\PYG{o}{.}\PYG{n}{cfg} \PYG{c}{\# buildout for production}
\PYG{n}{dev}\PYG{o}{.}\PYG{n}{cfg} \PYG{c}{\# buildout for development (extends buildout.cfg)}
\end{Verbatim}

\end{itemize}

\item {} 
Introduction to some generated binaries in bin/:
\begin{itemize}
\item {} 
\textbf{bin/ipython}: an interactive IPython shell with your project environment inside

\item {} 
\textbf{bin/zopepy}: an interactive Python shell with your project environment inside

\item {} 
\textbf{bin/paster}: The paster script to use for having the right PYTHONPATH

\item {} 
\textbf{bin/develop}: see \href{http://pypi.python.org/pypi/mr.developer}{mr.developer} documentation

\item {} 
\textbf{bin/zeoserver}: Zdaemon controler for the zeoserver if any

\end{itemize}

\item {} 
template initialization and starting your project:

\begin{Verbatim}[commandchars=@\[\]]
paster create -t minitage.plone31 myproject
cd /minitage/zope/myproject/ @&@& bin/instance start@textbar[]fg
bin/paster serve etc/CONFIG.ini @# alternative for WSGI mode
\end{Verbatim}

\end{itemize}


\subsubsection{Zope 3}
\label{paster/projects/model_projects:minitagezope3}\label{paster/projects/model_projects:zope-3}\begin{itemize}
\item {} 
template : \textbf{minitage.zope3}

\item {} 
minitage category : \textbf{zope}

\item {} 
Features
\begin{itemize}
\item {} 
Support for sob, relstorage and zeo mode

\item {} 
Support for wsgi

\item {} 
Support production and developement modes at the same time

\item {} 
Intuitive configuration files editing and regenration with buildout to
make them portables (zdaemon.conf.in zope.conf.in).

\end{itemize}

\item {} 
Layout additional infos:
\begin{itemize}
\item {} 
/category/project/zcml:

\begin{Verbatim}[commandchars=\\\{\}]
\PYG{n}{apidoc}\PYG{o}{.}\PYG{n}{zcml}\PYG{p}{,} \PYG{n}{site}\PYG{o}{.}\PYG{n}{zcml\PYGZus{}tmpl} \PYG{c}{\# ZCML files}
\end{Verbatim}

\item {} 
/category/project/etc:

\begin{Verbatim}[commandchars=@\[\]]
dev.ini, prod.ini @# WSGI configuration files
prod.zdaemon.conf.in @# zdaemon.conf buildout template for production
prod.zope.conf.in @# zope.conf buildout template for production
versions.cfg @# KGS file
zeo.conf.in @# zeo.conf buildout template
zdaemon.conf.in @# zdaemon.conf buildout template for dev mode
zope.conf.in @# zope.conf buildout template for dev mode
zeoserver.sh.in @# zeo launcher template
\end{Verbatim}

\end{itemize}

\item {} 
Introduction to some generated binaries in bin/:
\begin{itemize}
\item {} 
\textbf{bin/ipython}: an interactive IPython shell with your project environment inside

\item {} 
\textbf{bin/zopepy}: an interactive Python shell with your project environment inside

\item {} 
\textbf{bin/paster}: The paster script to use for having the right PYTHONPATH

\item {} 
\textbf{bin/develop}: see \href{http://pypi.python.org/pypi/mr.developer}{mr.developer} documentation

\item {} 
\textbf{bin/z3-ctl}: Zdaemon controller for production WSGI configuration

\item {} 
\textbf{bin/z3-dev}: Zdaemon controller for developement WSGI configuration

\item {} 
\textbf{bin/z3-app}: call your application in debug mode

\item {} 
\textbf{bin/zeoserver}: Zdaemon controler for the zeoserver if any

\end{itemize}

\item {} 
Another note on ZDaemon, it can be used as a init script for your application,
althought, you can use directly paster. Another way to make your application
``initd compliant'' is to use the \code{minitage.paste-initd} instancee.

\item {} 
You can regenerate the *.conf after editing the \code{.in} ine \code{etc/}:

\begin{Verbatim}[commandchars=@\[\]]
bin/buildout install @PYGZlb[]basename of config.file@PYGZrb[]
eg:
  vim etc/zeo.conf.in @# be careful, emacs is dangerous for your fingers health
  bin/buidout install zeo.conf
\end{Verbatim}

\item {} 
The Spawning ultra perforamnt WSGI server is only available with ZEO or RelStorage because of
its multithreading design.

\item {} 
template initialization and starting your project:

\begin{Verbatim}[commandchars=@\[\]]
paster create -t minitage.zope3 myproject
cd /minitage/zope/myproject/ @&@& bin/myproject-ctl restart
\end{Verbatim}

\end{itemize}


\subsubsection{Pylons projects}
\label{paster/projects/model_projects:minitagepylons}\label{paster/projects/model_projects:pylons-projects}\begin{itemize}
\item {} 
template : \textbf{minitage.pylons}

\item {} 
minitage category : \textbf{pylons}

\item {} 
template initialization:

\begin{Verbatim}[commandchars=@\[\]]
paster create -t minitage.pylons myproject
\end{Verbatim}

\item {} 
It will generate an egg in develop mode into:

\begin{Verbatim}[commandchars=@\[\]]
@$mt/pylons/myproject/src/myproject
\end{Verbatim}

\item {} 
Under the hood, it just uses \textbf{paster create -t pylons}. This comes from
Pylons itself.

\item {} 
to start your project:

\begin{Verbatim}[commandchars=@\[\]]
mt/pylons/myproject/bin/paster serve mt/pylons/myproject/src/myproject/development.ini
\end{Verbatim}

\item {} 
Dont forget to keep the egg into version control.

\end{itemize}


\subsubsection{Django projects}
\label{paster/projects/model_projects:django-projects}\label{paster/projects/model_projects:minitagedjango}\begin{itemize}
\item {} 
template : \textbf{minitage.django}

\item {} 
minitage category : \textbf{django}

\item {} 
We lauch django-admin startproject as a postinstall step to quickstart
your project.

\item {} 
We also create an egg for the sake of well-done packaging from start to end.

\item {} 
\textbf{GeoDjango support}, just answer \textbf{yes} to GIS support

\item {} 
We provide paste configuration files to launch your application in a WSGI envrionnement, \emph{please stop using runserver}!
\begin{itemize}
\item {} 
It's done through \href{http://pypi.python.org/pypi/dj.paste}{dj.paste} integration with the generated egg.

\end{itemize}

\item {} 
Layout additional infos:
\begin{itemize}
\item {} 
etc/

\begin{Verbatim}[commandchars=\\\{\}]
\PYG{p}{[}\PYG{n}{dev}\PYG{o}{\textbar{}}\PYG{n}{prod}\PYG{p}{]}\PYG{o}{.}\PYG{n}{ini} \PYG{c}{\# Paste configuration files}
\end{Verbatim}

\item {} 
src/yourproject/

\begin{Verbatim}[commandchars=@\[\]]
Your project egg directory
\end{Verbatim}

\item {} 
bin/:

\begin{Verbatim}[commandchars=@\[\]]
paster @# the paster script to starting your project
yourproject* @# Various wrappers to classical django utilities
\end{Verbatim}

\item {} 
/category/project/:

\begin{Verbatim}[commandchars=\\\{\}]
\PYG{n}{buildout}\PYG{o}{.}\PYG{n}{cfg} \PYG{c}{\# buildout for customizing and deploying your project}
\end{Verbatim}

\end{itemize}

\item {} 
template initialization and starting your project:

\begin{Verbatim}[commandchars=@\[\]]
@$mt/bin/paster create -t minitage.django myproject
@$mt/categ/project/bin/paster serve @$mt/categ/project/etc/CONFIG.ini
@# or
@$mt/categ/project/bin/project@_manage runserver @# may God kill you :)
\end{Verbatim}

\end{itemize}


\subsection{Minitage and instances}
\label{paster/instances/index::doc}\label{paster/instances/index:mr-developer}\label{paster/instances/index:minitage-and-instances}

\subsubsection{Abstract}
\label{paster/instances/index:abstract}\begin{itemize}
\item {} 
Minitage instances are the way to integrate applications in minitage projects or simply generate convenient templates with less efforts.

\item {} 
The goal is to make boilerplate to facilitate the use of prefixed applications or the intialization of databases instances to ease the admins job.

\item {} 
The main thing is to simulate a subsystem is a subdirectory \code{/sys} of a project, or creating an instance of something, \code{./somewhere}.

\item {} 
The optionnal minitage support is mostly recent for all templates and can lead to bugs, please report them.

\end{itemize}


\subsubsection{Basic layout}
\label{paster/instances/index:basic-layout}
\begin{Verbatim}[commandchars=@\[\]]
/bin
    instancename.binary
/etc/
    /logrotate.d
    /init.d
    /cron.d
/var/
    /var/data
    /var/run
    /var/log
\end{Verbatim}

A postgresql example:

\begin{Verbatim}[commandchars=@\[\]]
/bin
    MyDatabase.psql
/etc/
    /logrotate.d
        MyDatabase.logrotate
    /init.d
        MyDatabase.postgresql
    /cron.d
        MyDatabase.cron
/var/
    /var/data
        MyDatabase/
    /var/run
        MyDatabase.pid
    /var/log
        postgresql/
            MyDatabase.log
\end{Verbatim}


\subsubsection{Listing the available instances}
\label{paster/instances/index:listing-the-available-instances}
\begin{Verbatim}[commandchars=@\[\]]
@$mt/bin/paster create --list-templates @textbar[] grep minitage @textbar[] grep instances
\end{Verbatim}


\subsubsection{Installing a instance}
\label{paster/instances/index:installing-a-instance}
The instances are just specialized paster templates so you just have to apply a template to a project.

There are too ways to use one instance, inside or without minitage, just by \textbf{anssering yes or no} to the minitage presence question.

Examples:
\begin{quote}

We will apply the `minitage.env' instance to `myproject'
\begin{itemize}
\item {} 
inside minitage:

\begin{Verbatim}[commandchars=@\[\]]
paster create -t minitage.instances.env myproject inside@_minitage=yes
Will produce: @$mt/categ/project/myproject/sys/share/minitage/minitage.env
\end{Verbatim}

\item {} 
without minitage:

\begin{Verbatim}[commandchars=@\[\]]
paster create -t minitage.instances.env myproject inside@_minitage=no
Will produce: myproject/share/minitage/minitage.env
\end{Verbatim}

\end{itemize}
\end{quote}


\subsection{Available instances}
\label{paster/instances/index:available-instances}

\subsubsection{minitage.instances.env}
\label{paster/instances/instances:minitage-instances-env}\label{paster/instances/instances::doc}
Installation:

\begin{Verbatim}[commandchars=@\[\]]
paster create -t minitage.instances.env myproject
\end{Verbatim}

\textbf{This is maybe the most important instance.}

This instance will install a `minitage.env' shell script.

When you source this shell script, all the environnement of the project is pushed into the current environment.

This will enable \emph{PATH}, and  \emph{LD\_LIBRARY\_PATH} to be filled with all the dependencies subdirectories.

the \emph{LD\_LIBRARY\_PATH} setted by minitage.paste may interact with the rest of the system. if you get troubles, just unset it:

\begin{Verbatim}[commandchars=@\[\]]
unset LD@_LIBRARY@_PATH
\end{Verbatim}

For example, if your project has \emph{postgresql} as a dependency, sourcing \emph{sys/share/minitage/minitage.env} will give you \emph{psql} in the \emph{PATH}.

It will also register some environement variables like:
\begin{itemize}
\item {} 
\emph{MT}: the path to the minitage top directory

\item {} 
\emph{INS}: the path to the current project

\end{itemize}


\subsubsection{minitage.instances.postgresql}
\label{paster/instances/instances:minitage-instances-postgresql}
Installation:

\begin{Verbatim}[commandchars=@\[\]]
paster create -t minitage.instances.postgresql myproject
\end{Verbatim}

This instance will install a `postgresql database'.

When you answer to question, think that the `main user' is a system user.
\begin{itemize}
\item {} 
In \emph{sys/bin}, you ll have binaries which point to the created database.

\item {} 
In \emph{sys/etc/init.d}, you ll have an init script to start the database.

\item {} 
In \emph{sys/var/data/postgresql/database}, you ll have the database.

\item {} 
In \emph{sys/var/log/postgresql}, you ll have the logs.

\end{itemize}


\subsubsection{minitage.instances.varnish}
\label{paster/instances/instances:minitage-instances-varnish}
Installation:

\begin{Verbatim}[commandchars=@\[\]]
paster create -t minitage.instances.varnish myproject
\end{Verbatim}

This instance will install a varnish instance.
\begin{itemize}
\item {} 
In \emph{sys/bin}, you ll have wrappers for original varnish binaries pointing to your instance

\item {} 
In \emph{sys/etc/init.d}, you ll have an init script to start the cache.

\item {} 
In \emph{sys/etc/varnish/}, you ll have the VCL configuration file

\item {} 
In \emph{sys/var/data/varnish}, you ll have the cache file.

\end{itemize}


\subsubsection{minitage.instances.varnish2}
\label{paster/instances/instances:minitage-instances-varnish2}
Installation:

\begin{Verbatim}[commandchars=@\[\]]
paster create -t minitage.instances.varnish2 myproject
\end{Verbatim}

This instance will install a varnish2 instance.
\begin{itemize}
\item {} 
In \emph{sys/bin}, you ll have wrappers for original varnish binaries pointing to your instance

\item {} 
In \emph{sys/etc/init.d}, you ll have an init script to start the cache.

\item {} 
In \emph{sys/etc/varnish/}, you ll have the VCL configuration file

\item {} 
In \emph{sys/var/data/varnish}, you ll have the cache file.

\end{itemize}


\subsubsection{minitage.instances.paste-initd}
\label{paster/instances/instances:minitage-instances-paste-initd}
Installation:

\begin{Verbatim}[commandchars=@\[\]]
paster create -t minitage.instances.paste-initd myproject
\end{Verbatim}

This instance will install an initd script to wrap a paste configuration start throught paster serve (with appropriate project dependencies).
\begin{itemize}
\item {} 
In \emph{sys/etc/init.d}, you ll have an init script to start the project.

\end{itemize}


\subsubsection{minitage.instances.mysql}
\label{paster/instances/instances:minitage-instances-mysql}
Installation:

\begin{Verbatim}[commandchars=@\[\]]
paster create -t minitage.instances.mysql myproject
\end{Verbatim}

This instance will install a `mysql database'.

When you answer to question, think that the `main user' is a system user.
\begin{itemize}
\item {} 
In \emph{sys/bin}, you ll have binaries which point to the created database.

\item {} 
In \emph{sys/etc/init.d}, you ll have an init script to start the database.

\item {} 
In \emph{sys/var/data/mysql/database}, you ll have the database.

\item {} 
In \emph{sys/var/log/mysql}, you ll have the logs.

\end{itemize}


\subsubsection{minitage.instances.cas}
\label{paster/instances/instances:minitage-instances-cas}
Be sure to have tomcat installed, and moreover, you must have the JAVA\_HOME variable defined in your environment.
Many distrib's as debian don't set it by default.

\begin{Verbatim}[commandchars=@\[\]]
export JAVA@_HOME=/path/to/your/jdk
minimerge tomcat-6.0.20 (if alreazdy installed : minimerge -NRuUv tomcat-6.0.20)
\end{Verbatim}

Installation

\begin{Verbatim}[commandchars=@\[\]]
easy@_install minitage.paste.extras
@$EDITOR @$MT/minilays/*/yourproject @# add tomcat-6.0.20 to dependencies
paster create -t minitage.instances.cas myproject
\end{Verbatim}

This instance will install a tomcat isntance where is installed a CAS server on top of it
\begin{itemize}
\item {} 
In \emph{sys/bin}, you ll have wrappers for original binaries pointing to your instance

\item {} 
In \emph{sys/etc/init.d}, you ll have an init script to start the tomcat.

\item {} 
In \emph{sys/etc/cas.project/}, you ll have the configuration files

\item {} 
In \emph{sys/var/data/tomcat/cas}, you ll have the tomcat instance files.

\end{itemize}


\chapter{How can i contribute to its development}
\label{index:how-can-i-contribute-to-its-development}

\section{Develop}
\label{develop:develop}\label{develop::doc}

\subsection{Tools for developers}
\label{develop:tools-for-developers}\begin{itemize}
\item {} 
We live on  \href{https://git.minitage.org}{git repositories}. If you need commit access, it will be over ssh and you ll need to ask kiorky for an account.

\item {} 
Bug tracking system is hosted on our \href{https://www.minitage.org/trac}{Trac}

\item {} 
We also have a \href{https://gitweb.minitage.org}{gitweb}.

\item {} 
We host the eggs on \href{http://pypi.python.org}{Pypi package index}
\begin{itemize}
\item {} 
\href{http://pypi.python.org/pypi/minitage.core}{http://pypi.python.org/pypi/minitage.core}

\item {} 
\href{http://pypi.python.org/pypi/minitage.paste}{http://pypi.python.org/pypi/minitage.paste}

\item {} 
\href{http://pypi.python.org/pypi/minitage.recipe}{http://pypi.python.org/pypi/minitage.recipe}

\end{itemize}

\end{itemize}


\subsection{Installing the development version of minitage}
\label{develop:pypi-package-index}\label{develop:installing-the-development-version-of-minitage}
If you want to be bleeding edge and not have a 3 years old debianised
minitage. You can give a try to the egg in development mode.
Be aware that you must know what you are doing there !
\begin{itemize}
\item {} 
If you need to, fire your virtualenv
\begin{quote}

\begin{Verbatim}[commandchars=\\\{\}]
\PYG{n+nb}{source} \PYG{n+nv}{\$prefix}/bin/activate
\end{Verbatim}
\end{quote}

\item {} 
Get the sources
\begin{quote}

\begin{Verbatim}[commandchars=\\\{\}]
mkdir -p \textasciitilde{}/repos/minitage
\PYG{n+nb}{cd} \textasciitilde{}/repos/minitage
git clone http://git.minitage.org/git/minitage/eggs/minitage.recipe/
git clone http://git.minitage.org/git/minitage/eggs/minitage.core/
git clone http://git.minitage.org/git/minitage/eggs/minitage.paste/
git clone http://git.minitage.org/git/minitage/eggs/buildout.minitagificator
\end{Verbatim}
\end{quote}

\item {} 
Install/Reinstall minitage in develop mode
\begin{quote}

\begin{Verbatim}[commandchars=\\\{\}]
\PYG{n+nb}{source} \PYG{n+nv}{\$prefix}/bin/activate
\PYG{n+nb}{cd} \textasciitilde{}/repos/minitage/minitage.core
python setup.py develop
\PYG{n+nb}{cd} \textasciitilde{}/repos/minitage/minitage.recipe
python setup.py develop
\PYG{n+nb}{cd} \textasciitilde{}/repos/minitage/minitage.paste
python setup.py develop
\PYG{n+nb}{cd} \textasciitilde{}/repos/minitage/buildout.minitagificator
python setup.py develop
\end{Verbatim}
\end{quote}

\item {} 
Sync its packages (all its minilays in minitage vocabulary).

\textbf{This will initiate also all the minitage directories for the first run.}
\begin{quote}

\begin{Verbatim}[commandchars=\\\{\}]
\PYG{n+nb}{source} \PYG{n+nv}{\$prefix}/bin/activate
minimerge -s
\end{Verbatim}
\end{quote}

\end{itemize}



\renewcommand{\indexname}{Index}
\printindex
\end{document}
